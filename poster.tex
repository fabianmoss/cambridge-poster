% A poster using beamerposter and the gemini theme with epfl colors.
% Beamerposter will let you create a poster in a single beamer frame
% using the columns and blocks from the beamer class.
% The color theme is adapted from Gemini's MIT scheme.

% use beamer as documentclass
\documentclass[final]{beamer}

% use the beamerposter package to create a poster
\usepackage[scale=1.4,orientation=portrait,size=a0]{beamerposter}
\usepackage{csquotes}
\usepackage[backend=biber,giveninits=true,maxbibnames=99,bibencoding=utf8,style=numeric]{biblatex}
\addbibresource{references.bib}
% make references small
\renewcommand*{\bibfont}{\scriptsize}

% use the gemini theme.
\usetheme{gemini}
% use the epfl color theme, defining epflred, -darkgray,  -lightgray, -gray, and -blue
\usecolortheme{epfl}

\usepackage{graphicx}
\usepackage{tikz}
\usepackage{adjustbox}
\usepackage{qrcode}
\usepackage{wrapfig}
\usepackage{blindtext}
\usepackage{subcaption}

\title{Inferring Tonality from Note Distributions: \\ Why Models Matter}

\author{Fabian C. Moss\textsuperscript{\textasteriskcentered}, Martin Rohrmeier}

\institute{Digital and Cognitive Musicology Lab, École Polytechnique Fédérale de Lausanne}


\begin{document}

\begin{frame}[t]

  \begin{minipage}[t][.56\textheight][t]{\textwidth}

  \begin{columns}[t]
    \begin{column}{0.3\textwidth}
      \begin{block}{Background}
        \alert{Pitch-class statistics} in pieces correspond to mental representations of tonality \cite{Albrecht2013,Harasim2019,Krumhansl1982, Temperley2001}.

        \begin{figure}
          \centering
          \includegraphics[width=\textwidth]{img/templates}
        \end{figure}
      \end{block}

      \begin{block}{Model 1: Circle of Fifths}
        Use \alert{models of tonal pitch space} to reveal further regularities in pitch-class distributions \autocite{Harasim2019}.
        \begin{figure}
          \centering
          \includegraphics[width=.9\textwidth]{img/radars}
        \end{figure}
      \end{block}

    \end{column}

    \begin{column}{0.3\textwidth}

      \begin{block}{Model 2: Line of Fifths}
        Using \alert{spelled pitch classes} enables distinction between diatonic, chromatic, and enharmonic pieces \autocite{Gardonyi2002}.
        \begin{figure}
				\centering

				\begin{subfigure}{\textwidth} % width of left subfigure
					\includegraphics[width=\textwidth]{img/gmm_josquin.png}
				\end{subfigure}

				\vspace{1em} % here you can insert horizontal or vertical space

				\begin{subfigure}{\textwidth} % width of left subfigure
					\includegraphics[width=\textwidth]{img/gmm_bach.png}
				\end{subfigure}

				\vspace{1em} % here you can insert horizontal or vertical space

				\begin{subfigure}{\textwidth} % width of right subfigure
					\includegraphics[width=\textwidth]{img/gmm_schubert.png}
				\end{subfigure}
			\end{figure}
      \end{block}

			\begin{block}{Conclusion}
        The often implicit or unconscious \alert{modeling assumptions about tonal spaces}
        underlying both pitch-class distributions in musical pieces and cognitive schemata
        greatly affect research outcomes. Making these assumptions explicit
        as well as incorporating music-theoretical knowledge about the
        structure of tonal spaces broadens incorporates modeling as an essential part
        to the research on the history of tonality.

      \end{block}

    \end{column}

    \begin{column}{0.3\textwidth}

			\begin{block}{Model 3: Tonnetz}
        More general models of tonal space reveal further developments in tonality.
				\begin{figure}
				\centering
				\begin{subfigure}{\textwidth} % width of left subfigure
					\includegraphics[width=\textwidth]{img/josquin_tonnetz.png}
				\end{subfigure}
				\begin{subfigure}{\textwidth} % width of left subfigure
					\includegraphics[width=\textwidth]{img/bach_tonnetz.png}
				\end{subfigure}
				\begin{subfigure}{\textwidth} % width of right subfigure
					\includegraphics[width=\textwidth]{img/schubert_tonnetz.png}
				\end{subfigure}
			\end{figure}
      \end{block}

    \end{column}
  \end{columns}

\end{minipage}

\begin{minipage}[t][.3\textheight][t]{\textwidth}

	\begin{columns}
		\begin{column}{.6\textwidth}
		  \begin{block}{Historical Development}

		    \begin{figure}
		      \centering
		      \includegraphics[width=\textwidth]{img/fifth_widths}
		      % \caption{}
		      % \label{}
		    \end{figure}

		  \end{block}
		\end{column}

		\begin{column}{.3\textwidth}
			\begin{block}{References}
          \printbibliography
      \end{block}

      \begin{block}{Acknowledgements}

        \begin{wrapfigure}{l}{.3\textwidth}
          \includegraphics[width=.3\textwidth]{img/Logo_EPFL.pdf}
        \end{wrapfigure}

        \small
        This research is generously supported by the Latour Chair in Digital Musicology at EPFL.
      \end{block}
		\end{column}

\end{columns}
\end{minipage}

\end{frame} % End of the enclosing frame

\end{document}
